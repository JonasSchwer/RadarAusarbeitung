%KFT

\subsection{Kontinuierliche FT}

\begin{defn}[\cite{BurgHaf}*{Definition 8.3}]
    Sei $f: \R\to\R$ stückweise stetig und absolut integrierbar. Gilt für $\hat{f}: \R\to\R$ die Zuordnung
    \begin{displaymath}
        \hat{f}(s) = \frac{1}{2\pi} \int_{-\infty}^{\infty}  f(t) \ehoch{-{\imag}st} {\diff}t, \quad s\in\R ,
    \end{displaymath}
    so nennt man $\hat{f}$ die \emph{Fouriertransformierte von f}. Man schreibt für $\hat{f}$ auch $\calF f$. Der Operator $\calF$ heißt dann \emph{Fouriertransformation}.

\end{defn}

\begin{satz}[\cite{BurgHaf}*{Satz 8.1}]
    Ist $\hat{f}$ die Fourier-Transformierte von $f$, so gilt
    \begin{displaymath}
        f(t) = \int_{-\infty}^{\infty}  \hat{f}(s) \ehoch{{\imag}ts} {\diff}s, \quad t\in\R.
    \end{displaymath}
\end{satz}
\begin{proof}
    Siehe Quelle.
\end{proof}

\begin{satz}[\cite{BurgHaf}*{Satz 8.5}]
    Seien $f_1,f_2$ stetige, beschränkte und absolut integrierbare Funktionen in $\R$. Für deren Faltung $f_1 * f_2$ gilt:
    \begin{displaymath}
      \calF(f_1 * f_2) = \calF f_1 ~ \calF f_2.
    \end{displaymath}
\end{satz}
\begin{proof}
    Siehe Quelle.
\end{proof}

